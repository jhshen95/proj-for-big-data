\documentclass{article}
\usepackage{amsmath}
\usepackage{amsfonts}
\usepackage{amssymb}
\usepackage{hyperref}
\usepackage{algorithm}
\usepackage[noend]{algpseudocode}
\makeatletter
\def\BState{\State\hskip-\ALG@thistlm}
\makeatother

\renewcommand{\algorithmicrequire}{\textbf{Input:}}
\renewcommand{\algorithmicensure}{\textbf{Output:}}


\usepackage{cite}
\bibliographystyle{plain}

\title{Report of Project 1\\A Review of Convexified Convolutional Neural Network}
\author{Xiaodong Jia}

\begin{document}
\maketitle

\tableofcontents
\newpage

\section{Introduction}
In this review, we analyze a two layer convexified convolutional neural network(CCNN) based on \cite{zhang2016convexified}, make a review of the related works and use some new methods to solve this problem.
\section{A Brief Summary of the Main Ideas}
The work in \cite{zhang2016convexified} is mainly about convexifying a two-layer convolutional neural network. Before it there is the convexification of a two-layer deep neural network work in \cite{aslan2013convex} and \cite{aslan2014convex}. The idea 
\subsection{Convex Two Layer and Deep Learning}
A convolutional neural network(CNN) can be written as a function $f(x)$,
\[f:\mathbb{R}^{d_0}\rightarrow\mathbb{R}^{d_2},\] where $d_0$ is the dimension of input vectors $x$, $d_2$ is the number of classes.

Particularly, for a two layer convolutional neural network, the following form separates the trainable parameters and others,
\[f^A(x):=(\text{tr}(Z(x)A_1),...,\text{tr}(Z(x)A_{d_2})),\]
where $A$ denotes all trainable parameters, and $Z$ only depends on the inputs.

If the loss function $\mathcal{L}(f;y)$ is convex about $f$, $\mathcal{L}(f^A(Z))$ is convex about $A$. We can then solve the convex optimization problem
\begin{align*}
\widehat{A}\in\text{argmin}_{\|A\|_*\leq R}\tilde{\mathcal{L}}(A)
\end{align*}
where $\tilde{\mathcal{L}}(A)=\sum_{i=1}^n\mathcal{L}(f^A(x_n);y_n)$, $n$ is the size of mini-batch, $\|\cdot\|_*$ denotes the nuclear norm, and $R$ is a restriction. $\widehat{A}$ is then transformed to the corresponding parameters of the CNN. As a result, the original non-convex problem is tranformed to a convex one. 
\section{The Construction of A Two-layer CCNN}
A two-layer CCNN can be written as a function $f:\mathbb{R}^{d_0}\rightarrow\mathbb{R}^{d_2}$, which takes in a vector $x$ that is often the vector-representation of a picture. In the context of this review, the output $f(x)$ is a discrete distribution vector, i.e. the $k$th element $f_k(x)\in\left[0,1\right] $ denotes the probability of $x$ belonging to class $k$.


In a common explanation, the construction of $f$ can be written as follows.

The input vector, or picture, $x$, is first separated to $P$ \emph{patches}, which can be written as a function $z_p(x)\in\mathbb{R}^{d_1},1\leq p\leq P$.

Then, each patch is transformed to $r$ scalars, which can be written as $h_j(z_p)=\sigma(w_j^Tz_p),1\leq j\leq r$, where $w_j\in\mathbb{R}^{d_1}$, $\sigma:\mathbb{R}\rightarrow\mathbb{R}$ is in general a non-linear function. Each $h_j$ is known as a \emph{filter}.

Now we have $P\times r$ scalars. These scalars are finally summed together with weights, denoting as $\alpha_{k,j,p}$. The two-layer CNN can then be written as
\[f_k(x):=\sum_{j=1}^{r}\sum_{p=1}^{P}\alpha_{k,j,p}h_j(z_p(x)).\]
When $\sigma$ is identity, i.e. $\sigma(x)=x,x\in\mathbb{R}$, we can separate the trainable parameters $\alpha,w$ with other constants. Rewritten $f_k$ as
\begin{align*}
f_k(x)&=\sum_{j=1}^{r}\sum_{p=1}^{P}\alpha_{k,j,p}h_j(z_p(x))\\
&=\sum_{j=1}^{r}\sum_{p=1}^{P}\alpha_{k,j,p}w_j^Tz_p(x)\\
&=\sum_{j=1}^{r}\alpha_{k,j}^TZw_j\\
&=\sum_{j=1}^{r}\text{tr}(\alpha_{k,j}^TZw_j)\\
&=\sum_{j=1}^{r}\text{tr}(Zw_j\alpha_{k,j}^T)\\
&=\text{tr}(Z\sum_{j=1}^{r}w_j\alpha_{k,j}^T)\\
&=\text{tr}(ZA_k)
\end{align*}
where in the second equation $Z=(z_1(x),\cdots,z_P(x))^T$, in the third equation $\alpha_{k,j}=(\alpha_{k,j,1},\cdots,\alpha_{k,j,P})^T$ and in the final equation $A_k=\sum_{j=1}^{r}w_j\alpha_{k,j}^T$.

Thus, let $A:=(A_1(x),\cdots,A_{d_2}(x))$ denoting all $A_k$, so $A$ is in fact all the trainable parameters. We can then define a function \[f^A:=(\text{tr}(ZA_1),\cdots,\text{tr}(ZA_{d_2})),\] which is a linear function corresponding to $A$.

The CNN model class is a collection of such functions with constraints on $A$. In particular, we define
\[\mathcal{F_{\text{CNN}}}(B_1,B_2):=\lbrace f^A|\max_{j\in[r]}\|w_j\|_2\leq B_1, \max_{k\in[d_2],j\in[r]}\|\alpha_{k,j}\|_2\leq B_2,\text{rank}(A)=r \rbrace\]
where $[x]=\lbrace i|1\leq i\leq x\rbrace.$

So $\mathcal{F}(B_1,B_2)$ includes all such functions with a limited trainable parameters, and the rank constraint inherits from the formulation of $A_k$. Now, the matrix $A$ can be decomposed as $A=UV^T$, where both $U$ and $V$ have $r$ columns. The column space of $A$ contains the convolution parameters $\lbrace w_j\rbrace$, and the row space of $A$ contains the output parameters $\lbrace\alpha_{k,j}\rbrace$.

Now, the matrices $A$ satisfying the constraints in $\mathcal{F}$ in fact form a non-convex 	set. To make it convex, a standard relaxation is based on the nuclear norm $\|A\|_*$ which is the sum of the singular values of $A$. By the triangle inequality we have
\begin{align*}
\|A\|_*&=\|(A_1,\cdots,A_P)\|_*\\
&\leq\sum_{j=1}^r\|w_j(\alpha_{1,j}^T,\cdots,\alpha_{d_2,j}^T)\|_*\\
&\leq\sum_{j=1}^r\|w_j\|\|(\alpha_{1,j}^T,\cdots,\alpha_{d_2,j}^T)\|_*\\
&\leq rB_1\sqrt{d_2B_2^2}\\
&=r\sqrt{d_2}B_1B_2.
\end{align*}
Thus, we can define a more general CCNN class by
\[\mathcal{F}_{\text{CCNN}}:=\lbrace f^A|\|A\|_*\leq r\sqrt{d_2}B_1B_2\rbrace,\]
which is convex and we have $\mathcal{F}_{\text{CCNN}}\supseteq\mathcal{F}_{\text{CNN}}$.

Now, let $\mathcal{L}(f(x),y)$ denote the loss function, where $y$ is the label of $x$. We assume $\mathcal{L}$ is convex and $L$-Lipschitz in the first argument. Our aim is then compute
\[\widehat{f}_{CCNN}:=\text{argmin}_{f^A\in\mathcal{F}_{\text{CCNN}}}\sum_{i=1}^n\mathcal{L}(f^A(x_i);y_i).\]

When $\sigma$ is a non-linear function, we transform the filter $h_{\sigma,w}(z_p(x))$ to a linear form by
\[h(z_p(x_i))=\sum_{(i',p')\in[n]\times[P]}c_{i',p'}k(z_p(x_i),z_{p'}(x_{i'})),\]
where $k$ is a positive semi-definite kernel function. Viewing $k(*,z_{p'}(x_{i'})),(i',p')\in[n]\times[P]$ as a basis, the filter $h$ is represented by $c_{i',p'}$. Now, let $K\in\mathbb{R}^{nP\times nP}$,
\[K_{(i,p),(i',p')}=k(z_p(x_i),z_{p'}(x_{i'})).\]
Consider a factorization $K=QQ^T$, where $Q\in\mathbb{R}^{nP\times m}$, we can rewrite $h$ by 
\[h(z_p(x_i))=c^T(Q_{(i,p)}Q^T)^T=\langle Q_{(i,p)},c^TQ\rangle.\]
Let $w=c^TQ$, we have $h(z_p(x_i))=\langle Q_{(i,p)},w\rangle$. In order to learn the filter $h$, it suffices to learn $w$. The non-linear $h$ is thus transformed to a linear function(of $w$).\\
Now, as in the linear case but denoting $Q$ as $Z$, we have
\begin{align*}
f_k(x)&=\sum_{j=1}^{r}\sum_{p=1}^{P}\alpha_{k,j,p}h_j(z_p(x))\\
&=\text{tr}(QA_k)\\
&=\text{tr}(ZA_k).
\end{align*}
Suppose we have computed a result $A\in\mathbb{R}^{m\times Pd_2}$. At test time, given a new input $x\in\mathbb{R}^{d_0}$, we can compute the matrix $Z$ as follows.

Notice that the the $(i,p)$th row of $K$ corresponding to $z_p(x_i)$, denoted as $K_{(i,p)}$, is $K_{(i,p)}=Q_{(i,p)}Q^T$. Let $v(z_p(x))=(k(z_p(x),z_{p'}(x_{i'})))_{(i',p')}\in\mathbb{R}^{nP}$, we want to find the representation $c_x$ so that
\[c_x=\text{argmin}_{c_x}\|v(z_p(x))^T-c_x^TQ^T\|_2^2=\|v(z_p(x))-Qc_x\|_2^2.\]
The answer is $c_x=Q^\dagger v_p(x)$, where $v_p(x)=v(z_p(x))$. So we have $Z(x)=(Q^\dagger v(x))^T$.



\begin{algorithm}
\caption{Learning Two-layer Convexified Convolutional Neural Networks}\label{euclid}
\algorithmicrequire{ Data $\lbrace(x_i,y_i)\rbrace_{i=1}^n$, kernel function $\mathcal{K	}$, regularization parameter $R>0$, number of filters $r$.}
\begin{enumerate}
\item Construct a kernel matrix $K\in\mathbb{R}^{nP\times nP}$ such that the entry at column $(i,p)$ is and row $(i',p')$ is equal to $\mathcal{K}(z_p(x_i),z_{p'}(x_{i'}))$.
\item Compute a factorization $K=QQ^T$ or an approximation  $K\approx QQ^T$, where $Q\in\mathcal{R}^{nP\times m}$.
\item For each $x_i$, construct patch matrix $Z(x_i)=(Q^\dagger v(x_i))^T\in\mathbb{R}^{P\times m}$
\item Solve the following optimization problem to obtain a matrix $\widehat{A}=(\widehat{A}_1,\cdots,\widehat{A}_{d_2})$:
\[\widehat{A}\in\text{argmin}_{\|A\|_*\leq R}\tilde{\mathcal{L}}(A):=\sum_{i=1}^m\mathcal{L}(\text{tr}(Z(x_i)A),\cdots,\text{tr}(Z(x_i)A_{d_2});y_i).\]
\item Compute a rank-$r$ approximation $\tilde{A}\approx\widehat{U}\widehat{V}^T$ where $\widehat{U}\in\mathbb{R}^{m\times r}$ and $\widehat{V}\in\mathbb{R}^{Pd_2\times r}$.
\end{enumerate}
\algorithmicensure{ Return the predictor $\widehat{f}_{\text{CCNN}}(x):=(\text{tr}(Z(x)\widehat{A}_1),\cdots,\text{tr}(Z(x)\widehat{A}_{d_2}))$ and the convolutional layer output $H(x):=\widehat{U}^T(Z(x))^T$.}
\end{algorithm}


\bibliography{report_xdj.bib}
\end{document}